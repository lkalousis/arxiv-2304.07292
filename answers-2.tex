\documentclass[a4paper,11pt]{article}
\pdfoutput=1 

%\usepackage{jinstpub} 
%

\usepackage{enumitem} 
%

\begin{document}

{\bf Answers to the referee}
\\[1ex]

I would like to thank the referee for all his comments. 
I have tried to address them along the lines that now follow:
\\[1ex]

C: \emph{ I would suggest removing the type of PMT from the title, making it more generic. } 
	
A: I have changed the title to ``Calibration of photomultiplier tubes'' %with gaussian single photoelectron response'' 
to accommodate the referee's comment.
Several parts of the text (especially in the Introduction) have changed, but only lightly, to reflect this modification. 
\\[1ex]

C: \emph{The tests with PMT could still be mentioned in the abstract} 

A: I have added a mention in the abstract. Namely: 

This scheme was applied to the calibration of the Hamamatsu R7081 photomultiplier tube and a comparison of the DFT approach with the more standard numerical integration method is also presented.
Last, toy Monte Carlo were analyzed for different values of $\mu$ to understand the precision of the DFT method.
\\[1ex]

C: \emph{ It would be more evident if the author would provide a comparison of the new method with the original one and demonstrate the improvement of the precision. }
	
A: A subsection dedicated to this study has been added (subsection 3.3). 
The referee may have a look, but in a nutshell, inside the $\mu\sim$ 0.5 - 0.2 range both methods return the same gain. 
The advantages of the DFT lie in the following:
\begin{itemize}
\item DFT gives a better $\chi^2$/NDOF and, 
\item DFT runs much faster. 
\end{itemize}

C: \emph{Another point is the systematics attributed to the (improved or not) model of SPE. 
The tests with toy MC would be more informative with respect to the experimental tests.}

A: I have added a full section to cover this point (section 4). The referee can have a look, but I summarize here the main points: 
\begin{itemize}
\item The DFT provides a smaller bias in the extraction of the gain
\item It provides a better fit (it gives a better $\chi^2$/NDOF)
\item It runs much faster 
\end{itemize}
The last two points have been observed with real PMT data. 
The first one is treated here with the toy MC. 
In particular, the DFT approach is accurate at $\sim$ 0.3\% level for $\mu=5$. 
\\[1ex]

C: \emph{I recommend adding comparison of improved/original models both for the experimental data and for the toy MC.}

A: Have been treated above.
\\[1ex]

The Outlook has been slightly edited to account for these modifications. 
In particular the paragraph was added:
\\[1ex]

A comparison of the DFT and numerical integration methods was also attempted. 
In particular we showed, using R7081 data, that within the 0.5 -- 2.0 PE range both techniques provide consistent results, but DFT gives better $\chi^2$/NDOF and runs much faster. 
In this respect, we should point out that DFT is more appropriate for the analysis of large data samples. 
Several analysis were performed with toy Monte Carlo data for various values of $\mu$. 
Again, it was demonstrated that the DFT approach outperforms the numerical integration having an accuracy of better than 0.5~\% inside the $\mu\sim $ 0.5 -- 5.0 window. 
\\[1ex]

The paper (with the current changes) has been uploaded to JINST and the newest version has been submitted to arxiv (it should appear soon).

\end{document}
