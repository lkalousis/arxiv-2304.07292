\documentclass[a4paper,11pt]{article}
\pdfoutput=1 

%\usepackage{jinstpub} 
%

\usepackage{enumitem} 
%

\begin{document}

{\bf Answers to the referee}
\\[1ex]

Both referee's comments are (indeed) valid and we would like to thank him for bringing these to our attention. 
We have tried to address them in the following way: 
\\[1ex]

1. In all the measurements taken for the purposes of this publication the optical fiber was placed, using a halo-like plastic structure, 
at the center of the photocathode and the fiber was always pointing vertically with respect to the surface of the PMT.   
In this way, one has to deal with \emph{only one} quantum efficiency (that at the center of the PMT) and the possible bias 
described by the referee has to be absent. 

We should also point out that any remaining bias in the extraction of the gain, should be attributed to the SPE response model, 
eqs.~(3.1) and (3.4), and/or the stability of our setup. In particular, the SPE model proposed by Dossi \emph{et al.} is only an assumption 
and we might expect that it doesn't parametrize the charge response of the R7081 PMT in a rigorous way for all values of $\mu$. 
And even though the $\chi^2/NDOF$ returned by the minimizer was always close to one, 
one might expect that some bias can be induced in those circumstances where the validity of $S(x)$ ceases to apply. 

We have added the following paragraph in the body of this article, to address this point:
\\[1ex]

We should point out that for all the measurements included in this publication the optical fiber was placed at the center of the photocathode,
using a plastic halo-like structure, and that the fiber was always pointing vertically with respect to the surface of the PMT\footnote{%
More details on the experimental setup, including pictures of the halo-like structure, can be found in ref.~[12].}. 
In this way one has to deal a single quantum efficiency (that at the center of the PMT) and we expect the assumption of eq.~(2.1) to be quite valid. 
Note that the quantum efficiency of the PMT is expected to vary across the surface of the photocathode and a light source illuminating the whole surface of the PMT will not be 
accurately described by eq.~(2.1). In such cases one has to take into account the variance of the quantum efficiency across the incident angle and the Poissonian factors have to be modified. 
Note also that any possible bias in the extraction of the gain has to be attributed to the validity of the SPE response model (which was taken as an assumption)
and/or the stability of our setup. 




\newpage
2. We understand the referee's comment and we believe that it is a valid one.  
Nonetheless, in this article we have only tried to calibrate the R7081 PMT model in the just simplest case. 
When a single PMT is placed inside a dark and light-tight box, and with the optical fiber placed (vertically) at the center of the photocathode. 
The question of the \emph{in situ} calibration of large (monolithic) detectors and the associated accuracies is beyond the scope of this publication. 

We have added the following paragraph in the Outlook to reflect this view:
\\[1ex]

The following remarks are necessary. 
In this publication we have only tried to demonstrate the calibration of the R7081 PMT model in just the simplest case. 
That is, when a single PMT was placed inside a dark, light-tight box and with the optical fiber positioned at the center of the photocathode, and always pointing vertically towards the surface of the PMT. 
We have demonstrated that in this simple example, the gain remains remarkably stable (within $\sim 1\%$ or better) inside the $\mu \sim 0.5 - 2.0$ plateau. 
We should note that in large (monolithic) detectors equipped with a sizable number of PMTs this simplictic picture ceases to apply. 
In particular, in those circumstances we can expect the extraction of the gain to depend on the geometry of the detector and the position of the event. 
We can only expect that the accuracy achieved in this article will not be attainable in such cases. 
The question of the \emph{in situ} gain calibration of similar, complicated instruments lies beyond the scope of this article and was not treated here. 
More details on the energy and spatial resolution of large-volume liquid scintillator detectors cab be found elsewhere~[13]. 



\end{document}
